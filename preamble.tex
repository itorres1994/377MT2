%----------------------------------------------------------------------------------------
%	PACKAGES AND OTHER DOCUMENT CONFIGURATIONS
%----------------------------------------------------------------------------------------

\documentclass[a4paper,12pt,twoside]{book} % A4 paper and 11pt font size

\usepackage[utf8]{inputenc}
\usepackage[T1]{fontenc} % Use 8-bit encoding that has 256 glyphs
\usepackage[english]{babel} % English language/hyphenation
\usepackage{amsmath,amsfonts,amsthm} % Math packages
\usepackage{sectsty} % Allows customizing section commands
\usepackage{mathtools}
\usepackage{centernot}
\usepackage{color}
\usepackage{bbm}
\usepackage{amssymb}
\usepackage{enumerate}
\usepackage{fancyhdr} % Custom headers and footers
\usepackage{geometry}
\usepackage{mathpazo}
\usepackage{hyperref}
\usepackage{imakeidx}  % ADD INDEX
\usepackage{titlesec}  % WORK WITH CHAPTER/SECTION FORMATTING

\geometry{left=20mm,right=20mm,top=20mm}

% DOCUMENT WIDE SETTINGS
\raggedbottom
\makeindex[intoc]
\setcounter{secnumdepth}{1}           % Subsections are unnumbered
\newcounter{theoremnumber}[chapter]
\newcounter{asidenumber}[chapter]
\newcounter{problemcounter}[section]
\setlength{\parskip}{0.1em}

% Make subsection larger
\titleformat{\subsection}{\Large\bfseries\filcenter}{}{1.0em}{}

% SOME DEFINITIONS 
\def\R{\mathbb R}
\def\RS{\mathbb S}
\def\F{\mathbb F}
\def\Z{\mathbb Z}
\def\Q{\mathbb Q}
\def\C{\mathbb C}
\def\D{\mathbb D}
\def\H{\mathbb H}
\def\Re{\textbf{Re}}
\def\Im{\textbf{Im}}
\def\1{\mathbbm{1}}

%THEOREMS
\theoremstyle{definition}
\newtheorem*{prob}{\textbf{Problem :}}
\newtheorem*{prf}{Proof}
\newtheorem*{lem}{Lemma}
\newtheorem*{sln}{Solution}
\newtheorem*{fct}{Fact}
\newtheorem*{dfn}{Definition}
\newtheorem*{dfns}{Definitions}

\newtheorem{prop}[theoremnumber]{Proposition}
\newtheorem{cor}[theoremnumber]{Corollary}
\newtheorem{thm}[theoremnumber]{Theorem}
\newtheorem{aside}[asidenumber]{Aside}

\newtheorem{prb}[problemcounter]{Problem}

\newcommand{\irr}{\text{irr}}
\newcommand{\ol}[1]{\overline{#1}}
%NEW COMMANDS
\newcommand{\mc}[1]{\mathcal{#1}}
\newcommand{\mf}[1]{\mathfrak{#1}}
\newcommand{\bb}[1]{\mathbb{#1}}
\newcommand{\bbm}[1]{\mathbbm{#1}}
\newcommand{\ms}[1]{\mathscr{#1}}
\newcommand{\ttt}[1]{\texttt{#1}}
\newcommand{\includecode}[2][Python]{\lstinputlisting[caption=#2, escapechar=, style=custom#1]{#2}}
\newcommand{\embf}[1]{\textbf{\emph{#1}}}
\newcommand{\tbf}[1]{\textbf{#1}}
	% TOPOLOGY COMMANDS
\newcommand{\intr}[1]{\accentset{\circ}{#1}}
\newcommand{\bndr}{\partial}
\newcommand{\clsr}[1]{\overline{#1}}
\newcommand{\tpl}[1][]{\mathscr{T}_{#1}}
	% COMPLEX VARIABLE COMMANDS
\newcommand{\Log}{\text{Log}}
\newcommand{\partials}[2]{\frac{\partial #1}{\partial #2}}
\newcommand{\hessian}[3]{\left(\begin{array}{cc}\frac{\partial^2 #1}{\partial #2^2} & \frac{\partial^2 #1}{\partial #1 \partial #2} \\ \frac{\partial^2 u}{\partial #1\partial #2} & \frac{\partial^2 #1}{\partial #2^2}\end{array}\right)}
\newcommand{\Res}{\text{Res}}
    % LINEAR ALGEBRA COMMANDS
\newcommand{\Span}{\text{span}}
\newcommand{\Col} {\text{Col}}
\newcommand{\Row} {\text{Row}}
\newcommand{\Null}{\text{Null}}
\newcommand{\Rank}{\text{rank}}
\newcommand{\Mat}[2]{\,\text{Mat}_{{#1}\times{#2}}}
\renewcommand{\u}{\vec u}
\renewcommand{\v}{\vec v}
\newcommand{\w}{\vec w}
\newcommand{\x}{\vec x}
\newcommand{\y}{\vec y}
\newcommand{\z}{\vec z}
\newcommand{\im}{\text{im}}
    % ALGEBRA COMMANDS
\newcommand{\Stab}[2]{\,\text{Stab}_{#1}({#2})}
\newcommand{\Cent}[2]{\,\text{C}_{#1}({#2})}
\newcommand{\Center}[1]{\,\text{Z}({#1})}
\newcommand{\Norm}[2]{\,\text{N}_{#1}({#2})}
\newcommand{\subgp}{\leq}
\newcommand{\Orb}[1]{\,{#1}\text{-orbit}}
\newcommand{\orbit}[2]{\,\text{orbit}_{#1}({#2})}
\newcommand{\Inn}[1]{\,\text{Inn}({#1})}
\newcommand{\Aut}[1]{\,\text{Aut}({#1})}
\newcommand{\Syl}{\,\text{Syl}}
\newcommand{\normal}{\trianglelefteq}

\newcommand{\defn}[1]{\emph{#1}\index{#1}}
    % OTHER
\allsectionsfont{\centering \normalfont\scshape} % Make all sections centered, the default font and small caps

    % HYPERREF SETUP
\hypersetup{
    colorlinks,
    citecolor=black,
    filecolor=black,
    linkcolor=black,
    urlcolor=black
}


\pagestyle{fancy} % Makes all pages in the document conform to the custom headers and footers
\fancyhead[RO,LE]{\thepage} % No page header - if you want one, create it in the same way as the footers below
\fancyhead[LO]{\leftmark}
\fancyhead[RE]{\rightmark}
\fancyfoot{} % Empty left footer
\newcommand{\hrwidth}{0.8pt}
\newcommand{\chrwidth}{1.8pt}
\renewcommand{\headrulewidth}{1.3pt} % Remove header underlines
\renewcommand{\footrulewidth}{0.2pt} % Remove footer underlines
\setlength{\headheight}{30.6pt} % Customize the height of the header

\numberwithin{equation}{section} % Number equations within sections (i.e. 1.1, 1.2, 2.1, 2.2 instead of 1, 2, 3, 4)
\numberwithin{figure}{section} % Number figures within sections (i.e. 1.1, 1.2, 2.1, 2.2 instead of 1, 2, 3, 4)
\numberwithin{table}{section} % Number tables within sections (i.e. 1.1, 1.2, 2.1, 2.2 instead of 1, 2, 3, 4)

\setlength\parindent{0pt} % Removes all indentation from paragraphs - comment this line for an assignment with lots of text
\newcommand{\horrule}[1]{\rule{\linewidth}{#1}} % Create horizontal rule command with 1 argument of height
