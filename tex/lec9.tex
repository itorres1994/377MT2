\setcounter{chapter}{8}
\chapter{Monitors and Condition Variables}

\section{What's wrong with \defn{semaphors}?}
\begin{itemize}
    \item are shared global variables
    \item no linguistic connection between semaphores and data they control
    \item can be accessed from anywhere
    \item dual purposed (mutex and sched constraints)
    \item no guarantee of proper usage
\end{itemize}

Solution: use a higher level construct

\section{Monitors}
A monitor is similar to a class that ties data/operations and
synchornization together. \\

They differ from classes by guaranteeing mutual exlusion and requiring all data
to be private.

\begin{dfn} 
    \begin{enumerate}
        \item (From Wikipedia) A \defn{monitor} is a synchronization construct
            that allows threads to have both mutual exclusion and the ability to
            wait (block) for a certain condition to become true.
        \item (From slides) A \emph{monitor} is a defines a \emph{lock} and zero
            or more \emph{condition variables} for managing concurrent access to
            shared data.
            \begin{itemize}
                \item Monitors use a \emph{lock} to ensure that only a single
                    thread is active in the monitor at a given time
                \item The \emph{lock} also provides mutual exclusion for shared
                    data
                \item \emph{Condition variables} enable threads to go to sleep
                    inside the critical sections, by releasing their lock at the
                    same time it puts the thread to sleep

            \end{itemize}
    \end{enumerate}
\end{dfn}

Monitor Operations:
\begin{itemize}
    \item Encapsulates shared date to protect
    \item Acquires the mutex at start
    \item Operates on the shared data
    \item Temporarily release mutex if it can't complete
    \item Reqcquires the mutex when it can continue
    \item Releases the mutex at the end
\end{itemize}

\subsection{Implementing Monitors in Java}

It is simple to turn a Java class into a monitor:

\begin{itemize}
    \item Make all data private
    \item Make all methods synchronized (or at least the non-private ones)
\end{itemize}

\begin{verbatim}
class Queue{
    private data;     // queue data

    public void synchronized Add(Object item) {
        put item on queue;
    }

    public void synchronized Remove(){
        if (queue not empty){
            remove item;
            return item;
        }
    }
}
\end{verbatim}


\section{More on Monitors -- From Wikipedia}
An alternate definition of a monitor:
\begin{dfn}
    A \defn{monitor} is a thread-safe class, object, or module that uses
    wrappted mutual exclusion in order to safely allow acess to a method or
    viariable by more than one thread.
\end{dfn}

\emph{The defining characteristic of a monitor is that its methods are executed
with mutual exclusion.} Thus we have an

\textbf{Invariant:} at each point in time, at most one thread may be executing
any of a monitors methods.

Monitors were invented by Per Brinch Hansen and C. A. R. Hoare, and were first
implemented in Brinch Hansen's Concurrent Pascal language.

\subsection{Mutual Exclusion}
While a thread is executing a method of a thread-safe object, it is said to
\emph{occupy} the object by holding its mutex. Thread-safe objects are
implemented to enforce that at \emph{each point in time}, at most one thread may
occupy the object. When a thread calls a thread-safe object's method it mus
twait until no other thread is occupying the thread-safe object.

\begin{figure}[h]
    \begin{verbatim}
class Account {
  private lock myLock

  private int balance := 0
  invariant balance >= 0

  public method boolean withdraw(int amount)
     precondition amount >= 0
  {
    myLock.acquire()
    try {
      if balance < amount {
        return false
      } else {
        balance := balance - amount
        return true
      }
    } finally {
      myLock.release()
    }
  }

  public method deposit(int amount)
     precondition amount >= 0
  {
    myLock.acquire()
    try {
      balance := balance + amount
    } finally {
      myLock.release()
    }
  }
}
    \end{verbatim}
    \caption{An implementation of a class with mutual exclusion}
\end{figure}



\section{Condition Variables}
\textbf{Question:} How can we change \texttt{remove()} to wait until something
is on the queue?
\begin{itemize}
    \item Logically, we want to go to sleep inside the critical section.
    \item But if we hold on to the lock and sleep, then other threads cannot
        access shared queue, add an item to it, and wake up the sleeping thread.
    \item \textbf{THREAD COULD SLEEP FOREVER}
\end{itemize}

\textbf{Solution:} use condition variables
\begin{itemize}
    \item Condition variables enable a thread to sleep inside a critical section
    \item Any lock held by the thread is atomically released when the thread is
        put to sleep.
\end{itemize}

\subsection{Operations on Condition Variables}
\begin{dfn}
    A \defn{condition variable} is a queue of threads waiting for something
    inside a critical section.
\end{dfn}

Condition variables support three operations:
\begin{enumerate}
    \item \texttt{Wait()}\index{wait}: atomic (release lock, go to sleep). When the
        process wakes up it re-acquires the lock
    \item \texttt{Signal()}\index{signal}: wake up waiting thread, if one exists.
    \item \texttt{Broadcast()}\index{broadcast}: wake up all waiting threads.
\end{enumerate}

\textbf{Invariant:} a thread must hold the lock while doing condition variable
operations.

In java, we use \texttt{wait()} to give up the lock, \texttt{notify()} to signal
that the condition a thread is waiting on is satisfied, \texttt{notifyAll()} to
wake up all waiting threads. Effectively there is one condition variable per
object.

\begin{verbatim}
class Queue{
    private Object[] queue;    // queue data

    public void synchronized Add(Object item) {
        queue.append(item);
        notify();
    }

    public Object synchronized Remove(){
        while (queue.size == 0)
            wait();           // Give up lock and go to sleep

        remove and return item;
    }
}
\end{verbatim}


\subsection{More on Condition Variabiables -- From Wikipedia}
Often, mutual exclusion is not enough. Threads attempting an operation may need
to wait until some condition \(P\) holds true. Busy waiting (ie, \texttt{while
(not P) continue;} since mutual exclusion will stop any other thread from
updating \(P\). There are other solutions but they have shortfalls (loop and
release, for example).

A classic example is the Producer/Consumer problem.

\begin{figure}[h]
\begin{verbatim}
global RingBuffer queue; // A thread-unsafe ring-buffer of tasks.

// Method representing each producer thread's behavior:
public method producer(){
    while(true){
        task myTask=...; // Producer makes some new task to be added.
        while(queue.isFull()){} // Busy-wait until the queue is non-full.
        queue.enqueue(myTask); // Add the task to the queue.
    }
}

// Method representing each consumer thread's behavior:
public method consumer(){
    while(true){
        while (queue.isEmpty()){} // Busy-wait until the queue is non-empty.
        myTask=queue.dequeue(); // Take a task off of the queue.
        doStuff(myTask); // Go off and do something with the task.
    }
}

\end{verbatim}
    \caption{The classic Producer/Consumer problem --- this code has a serious
    problem in that accesses to the queue can be interrupted and interleaved
    with other threads accesses to the queue. In particular, the
    \texttt{queue.enqueue()} and \texttt{queue.dequeue()} methods will likely
    have instructions to update the queue's member variables such as size, start
    and end positions, etc.}
\end{figure}






\section{Mesa vs Hoare Style Monitors}

\textbf{Question:} What should happen when \texttt{signal()} is called?
\begin{itemize}
    \item \textbf{If there are no waiting threads:} No waiting threads
        \(\implies \) the signaler continues and the signal is lost (different
        from behavior with semaphores)
    \item \textbf{If there is a waiting thread:} one of the threads starts executing,
        others must wait.
\end{itemize}

\textbf{Mesa-style:} \emph{(Nachos, Java, most real OSs)}
\begin{itemize}
    \item The thread that signals keeps the lock (and thus the processor)
    \item The waiting thread waits for the lock
\end{itemize}

There are two main styles of Monitor implementations. 
\textbf{Hoare-style:} \emph{(Most textbooks)}
\begin{itemize}
    \item The thread that signals gives up the lock and the waiting thread gets
        the lock
    \item When the thread that was waiting and is no executing exits or waits
        again, it releases the lock back to the signaling thread.
\end{itemize}
